\def\kallithea{
  %
  I created an opensource Python clone of Bitbucket / GitHub that supports
  both Git and Mercurial. I improved hi-res graphics, added Font Awesome,
  integrated advanced Mercurial features, and designed obsolete changeset
  graphlogs.
  %
}

\def\mercurial{
  %
  I am a frequent contributor and maintain two extensions: hgsubversion and
  remotenames. I have a deep understanding of the template engine and
  changeset evolution.
  %
}

\IfSubStr*{\jobname}{vitae}{
  \def\macports{
    % 
    What started out with me submitting fixes and bug reports for mostly
    scientific and mathematics ports ended up with me being invited to the
    \href{http://macports.org}{MacPorts} team as a permanent member. I now
    maintain around 90 ports and have helped overhaul standardizing compiler
    variants, something particularly helpful in the scientific computing
    community because it provides a uniform compiler suite for desired packages.
    % 
  }
}{
  \def\macports{
    %
    I maintain around 90 ports and have helped overhaul standardizing compiler
    variants, something particularly helpful in the scientific computing
    community.
    %
  }
}

\IfSubStr*{\jobname}{vitae}{
  \def\cluster{
    % 
    Built, setup, and administrated the \dept's 128 node computing cluster with
    an Infiniband backbone for my advisor \xiaofan. This included configuring a
    job queue, installing custom MPI, writing documentation, and teaching
    research groups basics of parallel programming.
    % 
  }
}{
  \def\cluster{}
}

\IfSubStr*{\jobname}{vitae}{
  \def\petscdesc{
    % 
    Initially, my work focused on integrating \petsc~on older physics-based code
    for plasma fusion simulation. Learning about library maintenance (especially
    linking issues) helped me disentangle a particularly difficult knot of 64-bit
    integers and direct solvers (e.g. \texttt{MUMPS}, \texttt{SuiteSparse}, and
    \texttt{SuperLU}). Specifically, I fixed linking with
    \texttt{METIS}/\texttt{ParMETIS} and its dependents. As my pre-doc position
    went on, this work expanded to include work on non-linear solvers, a major
    refactoring of time-stepping methods (for implementing implicit-explicit
    methods), and mentoring a student to add more options for a configuration
    parser.
    % 
  }
}{
  \def\petscdesc{}
}

\IfSubStr*{\jobname}{vitae}{
  \def\lsmsaaa{
    % 
    Brought up-to-date this organization's technology and communication
    capabilities, thereby laying the groundwork for a successful campaign for an
    amendment of our by-laws (the first since this association's incarnation)
    that ended annual dues. Other duties include organizing teleconferences,
    interfacing with the LSMSA Foundation, and helping to lead reunion weekend.
    % 
  }
}{
  \def\lsmsaaa{}
}

\IfSubStr*{\jobname}{vitae}{
  \def\positions{
    {2013--now}/{Kallithea Leader}/{Kallithea Project}/{}/{\noexpand\kallithea},
    {2012--now}/{Mercurial Developer}/{Mercurial Project}/{}/{\noexpand\mercurial},
    {2012--now}/{MacPorts Team Member}/{MacPorts Project}/{}/{\noexpand\macports},
    {2010--2012}/{Computing Cluster Administrator for \xiaofan}/{\iit}/{\chicago}/{\noexpand\cluster},
    {2010--now}/{PETSc Team Member}/{\anl}/{\chicago}/{\noexpand\petscdesc},
    {2009--now}/{Vice President}/{\lsmsa{} Alumni Association}/{\nat}/{\noexpand\lsmsaaa}
  }
}{
  \def\positions{
    {2013--now}/{Kallithea Leader}/{Kallithea Project}/{}/{\noexpand\kallithea},
    {2012--now}/{Mercurial Developer}/{Mercurial Project}/{}/{\noexpand\mercurial},
    {2012--now}/{MacPorts Team Member}/{MacPorts Project}/{}/{\noexpand\macports},
    {2010--now}/{PETSc Team Member}/{\anl}/{\chicago}/{\noexpand\petscdesc},
  }
}

%%% Local Variables: 
%%% mode: latex
%%% TeX-master: "resume"
%%% End: 
