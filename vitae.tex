\documentclass[11pt,a4paper]{moderncv}
\moderncvtheme[blue]{classic}

% adjust the page margins
\usepackage[scale=0.8]{geometry}
\usepackage{tikz}
\usepackage[utf8]{inputenc}
\usepackage[super]{nth}
\usepackage{todonotes}

\usepackage{biblatex}
\addbibresource{publications.bib}

% required when changes are made to page layout lengths
\AtBeginDocument{\recomputelengths}

\newcommand{\lsmsa}{LA School for Math, Science, and the Arts}
\newcommand{\nat}{Natchitoches, LA}
\newcommand{\lsu}{Louisiana State University}
\newcommand{\iit}{Illinois Institute of Technology}
\newcommand{\dept}{Dept. of Mathematics}
\newcommand{\br}{Baton Rouge, LA}
\newcommand{\anl}{Argonne National Laboratory}
\newcommand{\chicago}{Chicago, IL}
\newcommand{\hong}{H.~Zhang}
\newcommand{\lois}{L.C.~McInnes}
\newcommand{\blaise}{B.~Bourdin}

\newcommand{\bout}{\texttt{BOUT++}}

% personal data
\firstname{Sean}
\familyname{Farley}
\title{Mathematician}
\address{124 W. Polk St}{Chicago, IL}
\phone{(209) SEANFAR}
\email{sean@farley.io}
\extrainfo{http://farley.io}
% \photo[72pt]{photo}

% \nopagenumbers{}


% ----------------------------------------------------------------------------------
% content
% ----------------------------------------------------------------------------------

\begin{document}
\maketitle

\section{Research Interests}
\label{sec:research_interests}

\cvline{}{Finite element methods, numerical linear algebra, scientific
  computing, preconditioning algorithms, distributed version control, design
  patterns, software organization}

\section{Education}
\label{sec:education}

\cventry{2010--present}{Ph.D in Mathematics}{\iit}{\chicago}{}{Thesis:
  Efficient numerical treatment for a boundary integral method on graphic
  processors.}
\cventry{2006--2009}{M.S. in Mathematics}{\lsu}{\br}{}{}
\cventry{2001--2006}{B.S. in Mathematics}{\lsu}{\br}{}{Minors:
  \emph{Physics} and \emph{Chinese}}
\cventry{Summer 2005}{LSU Abroad Program}{Beijing Normal University}{Beijing, China}{}{}
\cventry{1999--2001}{LSMSA Diploma}{Louisiana School for Math, Science, and the Arts}
  {\nat}{}{Distinction: Computer Science}

\section{Experience}
\label{sec:experience}

\cventry{6/2013--9/2013}{Google Summer of Code Student}{Mercurial Project}{\chicago}{}{
%
  Accepted to work on improving the record API of Mercurial. This was a very
  educational process of learning the depths of internal Mercurial code,
  specifically the context, manifest, and localrepo objects. The biggest hurdle
  of this project was tackling refactoring the localrepo.status() and
  localrepo.commit() functions. This remains a work-in-progress to date.
  \todo[inline]{Add blog posts and link here}
%
}

\cventry{1/2010--5/2013}{Predoctorate Research Assistant}{\anl}{\chicago}{}{
%
  Worked on adding preconditioning and new mesh capabilities to \bout. This has
  been released in version 0.85.
  \todo[inline]{This is the section that needs the most expansion}
  \todo[inline,color=green!40]{explain the preconditioning work (refer to publication)}
  \todo[inline,color=blue!40]{probably re-word mesh work as improvements in
    general and consider elaborating about the build system}
%
}

\cventry{Summer 2009}{Student Co-op}{\anl}{\chicago}{}{
%
  Continued work with \hong~and \lois~to build extendable code in collaboration
  with LLNL physicists by adding robust PETSc solvers to \bout, a parallel edge
  turbulence framework for plasma fusion simulation. This code has been
  successfully integrated into \bout{} and has been released in version 0.7.
  \todo[inline]{Definitely need to add equations and explain more about the
    physicist collaboration}
%
}

\cventry{8/2006--12/2009}{Graduate Assistant}{\lsu}{\br}{}{
%
  Assisted \blaise~in a graduate class on elliptic solvers and wrote supporting
  code in C/C++ using PETSc. Installation, configuration and administration of
  LDAP, http and svn servers for a cluster. General departmental computer
  support: design of a Kerberos-based authentication system.
  \todo[inline]{Probably should expand the PETSc and provide equations for
    elliptic solvers}
%
}

\cventry{Summer 2008}{Givens Internship}{\anl}{\chicago}{}{
%
  Worked under \hong~and \lois~to add PETSc time-stepping and preconditioners
  to a edge plasma simulation code from LLNL.
%
}

\cventry{8/2007--5/2008}{Undergraduate Math Teacher}{\lsu}{\br}{}{
%
  Taught $\sim$100 students a semester College Algebra and Trigonometry as the
  teacher of record. In addition to the standard responsibilities of teaching
  (creating lesson plans, grading, etc.), this course also required mandatory
  weekly math lab participation involving modern educational software and
  forward pedagogical methods.
%
}

\cventry{Summer 2006}{Research Programmer with \blaise}{\lsu}{\br}{}{
%
  Wrote parallel programs using MPI and PETSc; structuring data, scatter and
  gather among processors.
  \todo[inline]{Expand MPI and PETSc?}
%
}

\cventry{6/2003--8/2006}{.NET Programmer}{Velocity Squared, LLC}{\br}{}{
%
  Worked with small teams to solve numerous business and technical
  problems. Used design patterns, databases, threads, security, web services,
  and xml.
  \todo[inline]{Since this was my first ``real'' job, I feel like I should put
    more here \ldots but it was so long ago and my memory is fading}
%
}

\cventry{8/2001--7/2002}{PHP Programmer}{Louisiana Department of Education}{\br}{}{
%
  Designed online web applications with extensive database access using PHP and
  created graphics for web pages. Configured hardware, installed software, and
  taught workshops on Javascript, SQL, and Adobe Photoshop 6.
%
}

\cventry{8/2001--5/2002}{Data Visualization Programmer with J.~Tohline}{\lsu}{\br}{}{
%
  Worked with a general relativity research group that used OpenGL and CAVE to
  implement with C++ libraries to utilize 3D aspects of the SGI ImmersaDesk on
  modeling binary stars.
%
}

\cventry{8/2000--5/2001}{Dynamic Web Programmer}{\lsmsa}{\nat}{}{
%
  Designed and implemented web-based programs for students and faculty,
  utilizing server side languages such as PHP, Java Server Pages, and Perl. The
  main goal, though, was to complete an online application for registering
  students for seminars.
%
}

\cventry{8/1999--5/2000}{Web Programmer}{\lsmsa}{\nat}{}{
%
  Created web pages using HTML, CSS, and JavaScript for faculty, including
  graphic design and knowledge of web server maintenance.
%
}

\section{Honorary Positions}
\label{sec:honorary-positions}

\cventry{1/2012--present}{MacPorts Team Member}{MacPorts Project}{}{\chicago}{
%
  What started out as a fork of some scientific and mathematic ports, ended up
  with me being invited to the MacPorts team as a permanent member. I now
  maintain around 90 ports and have helped overhaul standardizing compiler
  variants, something particularly helpful in the scientific computing
  community.
%
}

\cventry{8/2010--5/2012}{Computing Cluster Administrator}{\iit}{\chicago}{}{
%
  Built, setup, and administrated the \dept's 128 node computing cluster with
  an Infiniband backbone. This included configuring a job queue, installing
  custom MPI, writing documentation, and teaching research groups basics of
  parallel programming.
%
}

\cventry{1/2010--present}{PETSc Team Member}{\anl}{\chicago}{}{
%
  The majority of my work started as library maintenance and integration for
  64-bit integers the graph partitioner \texttt{METIS} (e.g. \texttt{MUMPS},
  \texttt{SuiteSparse}, and \texttt{SuperLU}). As my pre-doc position went on,
  this work expanded to include some work on non-linear solvers, a major
  refactoring of time-stepping methods, and mentoring a student to add more
  options for a configuration parser.
%
}

\nocite{Dudson:2012wc}
\printbibliography[title={Publications}]

\section{Research Experience}
\label{sec:research_experience}

\subsection{Graduate Coursework}
\label{sub:graduate_coursework}

\def\courses{
  {Fluid Dynamics and Computational Methods}/{LSU},
  {Ordinary Differential Equations}/{LSU},
  {Finite Element Method: Analysis and Implementation}/{LSU},
  {Elliptic Solvers: Advanced parallel implementation \& related topics}/{LSU},
  {Fast Solvers}/{LSU},
  {Numerical Linear Algebra}/{LSU},
  {Applications of spectral theory material science}/{LSU},
  {Elliptic PDEs on non-smooth domains}/{LSU},
  {Finite Difference Methods}/{LSU},
  {Free discontinuity problems in image processing \& fracture mechanics}/{LSU},
  {Finite Volume Methods}/{IIT}}

\foreach \course/\school in \courses {
  \cvlistitem{\course, \emph{\dept}, \school}
}

\subsection{Awards}
\label{sub:awards}

\def\awards{
  {Fall 2009}/{NSF Fellowship under \blaise},
  {Summer 2008}/{Givens Fellowship under \lois},
  {2003, 2005}/{First place in Mathematics Association of America contest},
  {2001}/{First place in LSU sponsored computer programming competition},
  {}/{Graduated high school with distinction in computer science (A JSP/ODBC
    Client/Server Registration Package)},
  {2000,2001}/{Louisiana School Excellence Awards in Computer Science, Physics, and Mathematics}}

\foreach \year/\award in \awards {
  \cvline{\year}{\award}
}

\subsection{Organized Workshops}
\label{sec:workshops}

\def\workshops{
  {A Brief Overview of PETSc Capabilities that can be employed by \bout}/{\bout{} Workshop (Co-organizer)}/{Sept. \nth{14}, 2011},
  {Net-Centric Development: Building Enterprise Services}/{\lsmsa}/{May \nth{15} - May \nth{20}, 2011},
  {An Introduction to Parallel Programming with MPI}/{\lsmsa}/{Jan. \nth{4} - Jan. \nth{9}, 2010}}

\foreach \title/\place/\time in \workshops {
  \cvlistitem{``\title,'' \place, \time}
}

\subsection{Talks and Seminars}
\label{sub:talks_and_seminars}

\def\talks{
  {\bout: Performance Characterization and Recent Advances in Design}/{\nth{22} International Conference on Numerical Simulations of Plasma}/{Sept. \nth{8}, 2011},
  {\bout{} Current Status and Future Plans}/{FACETS Meeting}/{Feb. \nth{11}, 2010},
  {An Introduction to Ti\emph{k}Z: Integrating Graphics within \LaTeX}/{SIAM
    Student Seminar}/{Oct. \nth{30}, 2009},
  {\bout{} Status and Suggestions: A Look at Building Extendable
    Code}/{LLNL Physical and Life Sciences}/{Aug. \nth{17}, 2009},
  {Progress in Parallel Implicit methods for Tokamak Edge Plasma Modeling}/
  {H. Zhang, McInnes, T. Rognlien, M. Umansky and S. Farley, SIAM Conference on
    Computational Science and Engineering (CSE09), Miami, FL}/{March 2009},
  {Parallel Programming: An Introduction to PETSc and Distributed Arrays}/{LSU
    Math Seminar}/{Oct. \nth{30}, 2009},
  {Basic Parallel Programming: An Introduction to Linux and MPI}/{LSU Math
    Seminar}/{Oct. \nth{14}, 2009},
  {C Introduction: Pointers and Memory Allocation}/{LSU Math Seminar}/{Oct. \nth{1}, 2008}}

\foreach \title/\place/\time in \talks {
  \cvlistitem{``\title,'' \place, \time}
}

\section{Technical Skills}
\label{sec:technical_skills}

\todo[inline]{Eh, should I add more here?}

\cvlistitem{Over 5 years experience with parallel programming with MPI,
  including experience with CUDA, OpenCL, and some pthreads}

\cvlistitem{Over 7 years experience with building, maintaining, and programming
  on computing clusters}

\cvlistitem{Over 15 years experience in using open source projects such as
  Apache Web Server, MySQL, PHP, Python, Mercurial, MacPorts, and Linux / BSD
  kernels}

\cvlistitem{Over 15 years programming experience with C/C++, Java, Visual
  Basic, Perl, PHP, Python, JavaScript, HTML, SQL, Visual C++, C\#}

\cvlistitem{Compiled, setup, and maintained DHCP, DNS, LDAP, kerberos, database
  and email servers on FreeBSD and other *nix operating systems}

\cvlistitem{Extensive academic and work experience in computer related domains:
  scientific computing, data structures, design patterns, software design, web
  development, networking, etc.}

% \section{Extracurricular}
% \label{sec:extracurricular}

% \cvline{2007--2009}{SIAM Webmaster}
% \cvlistitem[-]{Responsible for setting up web server}
% \cvlistitem[-]{Maintained Wordpress}
% \cvlistitem[-]{Wrote custom scripts to maintain membership and calendar}
% \cvline{2005--2006}{Math Club -- President}
% \cvline{1994--1999}{Boy Scouts of America - Eagle Scout (4 palms)}

% \section{Miscellaneous}
% \label{sec:miscellaneous}
%
% \cvline{Citizenship:}{United States of America}
% \cvline{Marital Status:}{Unmarried}

\end{document}
