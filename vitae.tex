\documentclass[11pt,a4paper]{moderncv}
\moderncvtheme[blue]{classic}

% adjust the page margins
\usepackage[scale=0.8]{geometry}
\usepackage{tikz}
\usepackage[utf8]{inputenc}
\usepackage[super]{nth}
\usepackage{todonotes}

\usepackage{biblatex}
\addbibresource{publications.bib}

% required when changes are made to page layout lengths
\AtBeginDocument{\recomputelengths}

\newcommand{\lsmsa}{LA School for Math, Science, and the Arts}
\newcommand{\nat}{Natchitoches, LA}
\newcommand{\lsu}{Louisiana State University}
\newcommand{\iit}{Illinois Institute of Technology}
\newcommand{\dept}{Dept.\ of Mathematics}
\newcommand{\br}{Baton Rouge, LA}
\newcommand{\anl}{Argonne National Laboratory}
\newcommand{\chicago}{Chicago, IL}
\newcommand{\hong}{H.~Zhang}
\newcommand{\lois}{L.C.~McInnes}
\newcommand{\blaise}{B.~Bourdin}

\newcommand{\bout}{\texttt{BOUT++}}

% personal data
\firstname{Sean}
\familyname{Farley}
\title{Mathematician}
\address{124 W. Polk St}{Chicago, IL}
\phone{(209) SEANFAR}
\email{sean@farley.io}
\extrainfo{http://farley.io}
% \photo[72pt]{photo}

% \nopagenumbers{}


% ----------------------------------------------------------------------------------
% content
% ----------------------------------------------------------------------------------

\begin{document}
\maketitle

\section{Research Interests}
\label{sec:research_interests}

\cvline{}{Finite element methods, numerical linear algebra, scientific
  computing, preconditioning algorithms, distributed version control, design
  patterns, and software organization}

\section{Education}
\label{sec:education}

\cventry{2010--present}{Ph.D in Mathematics}{\iit}{\chicago}{}{Thesis:
  Efficient numerical treatment for a boundary integral method on graphic
  processors.}
\cventry{2006--2009}{M.S. in Mathematics}{\lsu}{\br}{}{}
\cventry{2001--2006}{B.S. in Mathematics}{\lsu}{\br}{}{Minors:
  \emph{Physics} and \emph{Chinese}}
\cventry{Summer 2005}{LSU Abroad Program}{Beijing Normal University}{Beijing, China}{}{}
\cventry{1999--2001}{LSMSA Diploma}{Louisiana School for Math, Science, and the Arts}
  {\nat}{}{Distinction: A JSP/ODBC Client/Server Multithreaded Registration Package.}

\section{Experience}
\label{sec:experience}

\cventry{6/2013--9/2013}{Google Summer of Code Student}{Mercurial Project}{\chicago}{}{
%
  Accepted to work to improve the recording feature (arbitrary selection of
  hunks upon committing) of Mercurial. This summer program provided a great
  educational process of learning the depths of internal Mercurial code,
  specifically the \texttt{context}, \texttt{manifest}, and \texttt{localrepo}
  objects. The biggest hurdle of this project was tackling the refactoring
  needed for the \texttt{status} and \texttt{commit} functions but I made
  substantial progress on both fronts.
%
}

\cventry{1/2010--5/2013}{Predoctorate Research Assistant}{\anl}{\chicago}{}{
%
  Added framework for physics-based preconditioning, new mesh capabilities, and
  advanced time-stepping methods to \bout~(BOUndary Turbulence), a parallel
  edge turbulence framework for plasma fusion simulation. Also improved the
  build system and software organization. The majority of work was improving
  meshing capabilities to add matrix preallocation for the non-contiguous
  branch-cuts. In addition, robust and novel IMEX (implicit-explicit)
  time-stepping methods were added giving the project competitive performance
  results. Version 1.0 incorporates my work. \cite{Dudson:2012wc} }

\cventry{Summer 2009}{Student Co-op}{\anl}{\chicago}{}{
%
  Continued work with \hong~and \lois~to build extendable code in collaboration
  with LLNL physicists by adding robust PETSc solvers to \bout. This code has been
  successfully integrated into \bout~and version 0.7 incorporates my work.
%
}

\cventry{8/2006--12/2009}{Graduate Assistant}{\lsu}{\br}{}{
%
  Assisted \blaise~ by writing code, providing proofs, building and maintaining
  a cluster, and helping with some class work. One such class was a graduate
  class on elliptic solvers ($\Delta u = 0$ for convex domains) using our
  previous finite element code from the summer. This code was extended to work
  on solving edge detection problems using the Mumford-Shah functional modeled
  withe phase fields. Other duties included helping the \dept's computer
  support team in the design of a Kerberos-based authentication system for use
  with LDAP, http, and Subversion authentication.
%
}

\cventry{Summer 2008}{Givens Internship}{\anl}{\chicago}{}{
%
  Worked under \hong~and \lois~to add PETSc preconditioners to an edge plasma
  simulation code written in C++ called \bout~from LLNL.
%
}

\cventry{8/2007--5/2008}{Undergraduate Math Teacher}{\lsu}{\br}{}{
%
  Taught around 100 students a semester College Algebra and Trigonometry as the
  teacher of record. In addition to the standard responsibilities of teaching
  (creating lesson plans, grading, etc.), my duties included assisting students
  during weekly mandatory math labs that employed online educational software.
%
}

\cventry{Summer 2006}{Research Programmer with \blaise}{\lsu}{\br}{}{
%
  Wrote a finite difference and a $P_1$, $P_2$ finite element program for
  solving elliptic equations using MPI and PETSc. The finite element code used
  a custom implementation based on PETSc distributed arrays. Key concepts were:
  structuring data, scatter / gather among processors, and linear solvers.
%
}

\cventry{6/2003--8/2006}{.NET Programmer}{Velocity Squared, LLC}{\br}{}{
%
  Worked with small teams to solve numerous business and technical problems
  with a photography and photo printing company. The main product would
  schedule regional head photographers allowing them to hire local
  photographers and coordinate shooting a local event. Our product would also
  gather all photos taken and upload them to the company's website, which
  allowed photographers to do some editing of each photo and then automatically
  putting selected photos into a product for sale. Used design patterns,
  databases, threads, security, web services, and xml.
%
  \newline\newline
%
  During the events of Hurricane Katrina, our entire team volunteered to aid
  the Red Cross with its difficult problem of keeping track of refugee victims
  across the city of Baton Rouge. After a 72-hour coding sprint, we had a
  functional program but due to legal constraints nothing was ever deployed
  full scale.
%
}

\cventry{8/2001--7/2002}{PHP Programmer}{Louisiana Department of Education}{\br}{}{
%
  Designed online web applications with extensive database access using PHP and
  created graphics for web pages. Other duties included configuring hardware,
  installing software, and teaching workshops for educators on Javascript, SQL,
  and Adobe Photoshop 6.
%
}

\cventry{8/2001--5/2002}{Data Visualization Programmer with J.~Tohline}{\lsu}{\br}{}{
%
  Worked with a general relativity research group that used OpenGL and CAVE to
  implement with C++ libraries to utilize 3D aspects of the SGI ImmersaDesk on
  modeling binary stars.
%
}

\cventry{8/2000--5/2001}{Dynamic Web Programmer}{\lsmsa}{\nat}{}{
%
  Designed and implemented web-based programs for students and faculty,
  utilizing server side languages such as PHP, Java Server Pages, and Perl. My
  most important focus area was completing my distinction: an online
  application to register students for seminars.
%
}

\cventry{8/1999--5/2000}{Web Programmer}{\lsmsa}{\nat}{}{
%
  Created web pages using HTML, CSS, and JavaScript for faculty. Other duties
  included graphic design and knowledge of web server maintenance.
%
}

\section{Honorary Positions}
\label{sec:honorary-positions}

\cventry{1/2012--present}{MacPorts Team Member}{MacPorts Project}{}{\chicago}{
%
  What started out with me submitting fixes and bug reports for mostly
  scientific and mathematics ports, ended up with me being invited to the
  MacPorts team as a permanent member. I now maintain around 90 ports and have
  helped overhaul standardizing compiler variants, something particularly
  helpful in the scientific computing community.
%
}

\cventry{8/2010--5/2012}{Computing Cluster Administrator}{\iit}{\chicago}{}{
%
  Built, setup, and administrated the \dept's 128 node computing cluster with
  an Infiniband backbone. This included configuring a job queue, installing
  custom MPI, writing documentation, and teaching research groups basics of
  parallel programming.
%
}

\cventry{1/2010--present}{PETSc Team Member}{\anl}{\chicago}{}{
%
  Initially, my work focused on integrating PETSc on older physics-based code
  for plasma fusion simulation. Learning about library maintenance (especially
  linking issues) helped me solve a particular difficult knot of 64-bit
  integers and direct solvers (e.g. \texttt{MUMPS}, \texttt{SuiteSparse}, and
  \texttt{SuperLU}, etc.). Specifically, I fixed linking with \texttt{METIS} /
  \texttt{ParMETIS} and its dependents. As my pre-doc position went on, this
  work expanded to include some work on non-linear solvers, a major refactoring
  of time-stepping methods (for implementing implicit-explicit methods), and
  mentoring a student to add more options for a configuration parser.
%
}

\cventry{5/2009--present}{Vice-President}{\lsmsa{} Alumni Association}{\nat}{}{
%
  Helped bring to this organization up-to-date with technology and
  communication, which in turn helped pass an amendment (the first since this
  association's incarnation) to forever get rid of yearly dues. Other duties
  include organizing teleconferences, interfacing with the LSMSA Foundation,
  and helping to lead reunion weekend.
%
}

\printbibliography[title={Publications}]

\section{Research Experience}
\label{sec:research_experience}

\subsection{Graduate Coursework}
\label{sub:graduate_coursework}

\def\courses{
  {Fluid Dynamics and Computational Methods}/{LSU},
  {Ordinary Differential Equations}/{LSU},
  {Finite Element Method: Analysis and Implementation}/{LSU},
  {Elliptic Solvers: Advanced parallel implementation \& related topics}/{LSU},
  {Fast Solvers}/{LSU},
  {Numerical Linear Algebra}/{LSU},
  {Applications of spectral theory material science}/{LSU},
  {Elliptic PDEs on non-smooth domains}/{LSU},
  {Finite Difference Methods}/{LSU},
  {Free discontinuity problems in image processing \& fracture mechanics}/{LSU},
  {Finite Volume Methods}/{IIT}}

\foreach \course/\school in \courses {
  \cvlistitem{\course, \emph{\dept}, \school}
}

\subsection{Awards}
\label{sub:awards}

\def\awards{
  {Fall 2009}/{NSF Fellowship under \blaise},
  {Summer 2008}/{Givens Fellowship under \lois},
  {2003, 2005}/{First place in Mathematics Association of America contest},
  {2001}/{First place in LSU sponsored computer programming competition},
  {}/{Graduated high school with distinction in computer science (A JSP/ODBC
    Client/Server Registration Package)},
  {2000,2001}/{Louisiana School Excellence Awards in Computer Science, Physics, and Mathematics}}

\foreach \year/\award in \awards {
  \cvline{\year}{\award}
}

\subsection{Organized Workshops}
\label{sec:workshops}

\def\workshops{
  {A Brief Overview of PETSc Capabilities that can be employed by \bout}/{\bout{} Workshop (Co-organizer)}/{Sept. \nth{14}, 2011},
  {Net-Centric Development: Building Enterprise Services}/{\lsmsa}/{May \nth{15} - May \nth{20}, 2011},
  {An Introduction to Parallel Programming with MPI}/{\lsmsa}/{Jan. \nth{4} - Jan. \nth{9}, 2010}}

\foreach \title/\place/\time in \workshops {
  \cvlistitem{``\title,'' \place, \time}
}

\subsection{Talks and Seminars}
\label{sub:talks_and_seminars}

\def\talks{
  {\bout: Performance Characterization and Recent Advances in Design}/{\nth{22} International Conference on Numerical Simulations of Plasma}/{Sept. \nth{8}, 2011},
  {\bout{} Current Status and Future Plans}/{FACETS Meeting}/{Feb. \nth{11}, 2010},
  {An Introduction to Ti\emph{k}Z: Integrating Graphics within \LaTeX}/{SIAM
    Student Seminar}/{Oct. \nth{30}, 2009},
  {\bout{} Status and Suggestions: A Look at Building Extendable
    Code}/{LLNL Physical and Life Sciences}/{Aug. \nth{17}, 2009},
  {Progress in Parallel Implicit methods for Tokamak Edge Plasma Modeling}/
  {H. Zhang, McInnes, T. Rognlien, M. Umansky and S. Farley, SIAM Conference on
    Computational Science and Engineering (CSE09), Miami, FL}/{March 2009},
  {Parallel Programming: An Introduction to PETSc and Distributed Arrays}/{LSU
    Math Seminar}/{Oct. \nth{30}, 2009},
  {Basic Parallel Programming: An Introduction to Linux and MPI}/{LSU Math
    Seminar}/{Oct. \nth{14}, 2009},
  {C Introduction: Pointers and Memory Allocation}/{LSU Math Seminar}/{Oct. \nth{1}, 2008}}

\foreach \title/\place/\time in \talks {
  \cvlistitem{``\title,'' \place, \time}
}

\section{Technical Skills}
\label{sec:technical_skills}

\cvlistitem{Over 5 years experience with analysis (real, complex, functional,
  numerical, partial differential equations, finite element, and some measure
  theory) and modeling (Cahn-Hilliard, Ideal Magnetic-hydrodynamics, Stokes Flow,
  and Mumford-Shah) equations}

\cvlistitem{Over 5 years experience with writing numerical code with PETSc,
  emphasizes on finite elements, difference, and volume methods; this includes
  intimate knowledge of numerical linear algebra, robust solvers,
  preconditioning, etc.}

\cvlistitem{Over 5 years experience with parallel programming with MPI,
  including experience with CUDA, OpenCL, and some pthreads}

\cvlistitem{Over 7 years experience with building, maintaining, and programming
  on computing clusters}

\cvlistitem{Over 10 years experience in \LaTeX{} with strong emphasis on Ti\emph{k}Z}

\cvlistitem{Over 15 years experience in using open source projects such as
  Apache Web Server, MySQL, PHP, Python, Mercurial, MacPorts, and Linux / BSD
  kernels}

\cvlistitem{Over 15 years programming experience with C/C++, Java, Visual
  Basic, Perl, PHP, Python, JavaScript, HTML, SQL, Visual C++, C\#}

\cvlistitem{Compiled, setup, and maintained DHCP, DNS, LDAP, kerberos, database
  and email servers on FreeBSD and other *nix operating systems}

\cvlistitem{Extensive academic and work experience in computer related domains:
  scientific computing, data structures, design patterns, software design, web
  development, networking, etc.}

\end{document}
