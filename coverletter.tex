%!TEX TS-program = xelatex
\documentclass[]{fancy-cv}
\hypersetup {
  colorlinks=true,
  urlcolor=bluepurple
}

\usepackage{tikz}
\usepackage[super]{nth}
\usepackage{todonotes}
\usepackage{xstring}
\usepackage{loops}[2013/05/01]

%%% Local Variables: 
%%% mode: latex
%%% TeX-master: "vitae"
%%% End: 

\newcommand{\lsmsa}{Louisiana School for Math, Science, and the Arts}
\newcommand{\nat}{Natchitoches, LA}
\newcommand{\lsu}{Louisiana State University}
\newcommand{\iit}{Illinois Institute of Technology}
\newcommand{\dept}{Dept.\ of Mathematics}
\newcommand{\br}{Baton Rouge, LA}
\newcommand{\anl}{Argonne National Laboratory}
\newcommand{\chicago}{Chicago, IL}
\newcommand{\sanfran}{San Francisco, CA}
\newcommand{\hong}{H.~Zhang}
\newcommand{\lois}{L.C.~McInnes}
\newcommand{\blaise}{B.~Bourdin}
\newcommand{\xiaofan}{X.~Li}

\newcommand{\petsc}{\href{http://mcs.anl.gov/petsc}{\texttt{PETSc}}}
\newcommand{\bout}{\href{http://www-users.york.ac.uk/~bd512/bout}{\texttt{BOUT++}}}
\newcommand{\mpi}{\href{http://www.mpich.org}{\texttt{MPI}}}
\newcommand{\Csharp}{%
  {\settoheight{\dimen0}{C}C\kern-.05em \resizebox{!}{\dimen0}{\raisebox{\depth}{\#}}}}

% biblatex settings
\addbibresource{bibliography.bib}

% Consider the whole date (year-month-day)
% Sorting date (descending), name, title
\DeclareSortingScheme{ddnt}{
  \sort{
    \field{presort}
  }
  \sort[final]{
    \field{sortkey}
  }
  \sort[direction=descending]{
    \field[strside=left,strwidth=4]{sortyear}
    \field[strside=left,strwidth=4]{year}
    \literal{9999}
  }
  \sort[direction=descending]{
    \field[padside=left,padwidth=2,padchar=0]{month}
    \literal{00}
  }
  \sort[direction=descending]{
    \field[padside=left,padwidth=2,padchar=0]{day}
    \literal{00}
  }
  \sort{
    \field{sortname}
    \field{author}
    \field{editor}
    \field{translator}
    \field{sorttitle}
    \field{title}
  }
  \sort{
    \field{sorttitle}
    \field{title}
  }
}
%%% Local Variables: 
%%% mode: latex
%%% TeX-master: "resume"
%%% End: 

\newcommand{\myfirstname}{Sean}
\newcommand{\mylastname}{Farley}
\newcommand{\mytitle}{Mathematician}
\newcommand{\myaddress}{124 W. Polk St}
\newcommand{\mycity}{Chicago, IL}
\newcommand{\myphone}{(209) SEANFAR}
\newcommand{\myemail}{sean@farley.io}
\newcommand{\mywebsite}{http://farley.io}
\newcommand{\myfb}{seanfarley}
\newcommand{\mygp}{+SeanFarley}

%%% Local Variables: 
%%% mode: latex
%%% TeX-master: "vitae"
%%% End: 


\begin{document}
\header{\MakeLowercase\myfirstname}{\MakeLowercase\mylastname}{software engineer}

\begin{aside}
  \section{contact}
    \href{mailto:\myemail}{\myemail}
    ~
    \myphone
    ~
    \myaddress
    \mycity
  \section{website}
    \href{\mywebsite}{\mywebsite}
  \section{social}
    \href{http://facebook.com/\myfb}{fb://\myfb}
    \href{http://plug.google.com/\mygp}{google://\mygp}
    \href{http://www.linkedin.com/in/\mylinkedin}{linkedin://\mylinkedin}
\end{aside}

\vspace{.8em}
\large
\today\\
\\
Facebook Human Resources\\
\\
\\
Dear Facebook Hiring Manager,

My mentor Matt Mackall, project leader of Mercurial, recommended that I forward
my resume to you for a position on the Mercurial team. As I finish my PhD in
Applied Mathematics, I am seeking a full-time position and I feel that I’d be
an excellent candidate for it.
%
\newline\newline
%
My unique background in computational mathematics and knowledge of Mercurial
internals has prepared me well for this position. My biggest accomplishment to
date is being a frequent contributor to the Mercurial project. I am one of the
few people who have submitted patches to changeset evolution in Mercurial and
care deeply about its development.
%
\newline\newline
%
Over the last two years, I accomplished a major overhaul of the in-memory
changeset, helped improve bookmark with obsoleting behavior, extended the debug
shell to work with ipython, and added color functions to the templater. These
have all taught me how to structure my code for submission and work with an
open-source community. My familiarity with the inner workings of Mercurial and
continual contribution to the community with Mercurial extensions, IRC
participation, and advocacy will help me succeed at integrating with the
Facebook Mercurial team.
%
\newline\newline
%
As a predoctorate at Argonne National Laboratory, I worked in the
\texttt{PETSc} group of computational scientists, primarily using C and
Python. My work at ANL has prepared me for real-life applications of
large-scale operations. For example, I solved partial differential equations
involving large matrices and graphs \& worked with parallel filesystems.
%
\newline\newline
%
I am eager to pursue this opportunity. The best way to contact me to set up an
interview is by email. For further information about my commit history, please
visit my website. I look forward to connecting with you.

\vspace{2em}
Sincerely,\\
\\
Sean Farley

\end{document}
