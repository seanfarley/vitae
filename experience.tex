\IfSubStr*{\jobname}{vitae}{
  \def\gsoc{
    %
    \setlength\parindent{24pt} \emph{Engineered deep refactor of the internal
      representation of changesets in memory}.
    \\
    This work opened the way for improving the recording feature (arbitrary
    selection of hunks upon committing) and in-memory rebasing for
    \href{http://mercurial.selenic.com}{Mercurial}.  The biggest hurdle of this
    project was tackling the refactoring needed for the
    \href{http://selenic.com/hg/file/c38c3fdc8b93/mercurial/localrepo.py\#l1458}{\texttt{status}}
    and
    \href{http://selenic.com/hg/file/c38c3fdc8b93/mercurial/localrepo.py\#l1149}{\texttt{commit}}
    functions. This summer program provided a great educational process of
    learning the depths of internal Mercurial code, specifically the
    \href{http://selenic.com/hg/file/tip/mercurial/context.py}{\texttt{context}},
    \href{http://selenic.com/hg/file/tip/mercurial/manifest.py}{\texttt{manifest}},
    and
    \href{http://selenic.com/hg/file/tip/mercurial/localrepo.py}{\texttt{localrepo}}
    objects. This experience was successfully completed with over 127 patches
    accepted.
    %
  }
}{
  \def\gsoc{
    %
    \vspace{-4mm}
    \begin{itemize}
      \item Engineered deep refactor of representation of changesets in memory
      \item Extracted classes to improve in-memory commit and rebase
      \item Received a successful evaluation with over 127 patches accepted
    \end{itemize}
    %
  }
}

\IfSubStr*{\jobname}{vitae}{
  \def\predoc{
    %
    \setlength\parindent{24pt} \emph{Wrote a paper on \bout~(BOUndary
      Turbulence), a parallel edge turbulence framework.}
    \\
    I added framework for physics-based preconditioning, advanced nonlinear
    solvers, advanced time-stepping methods, new mesh capabilities, and
    improved the build system for better software organization. The majority of
    the work involved improving meshing capabilities to add matrix
    preallocation for the non-contiguous branch-cuts. In addition, robust and
    novel IMEX (implicit-explicit) time-stepping methods were added giving the
    project competitive performance results. Version 1.0 incorporates my work.
    %
  }
}{
  \def\predoc{
    %
    \vspace{-4mm}
    \begin{itemize}
      \item Published paper on \bout, a framework for plasma fusion simulation
      \item Developed advanced nonlinear solvers
      \item Integrated robust time-stepping methods from \petsc
      \item Added framework for new mesh capabilities
      \item Devised an approximate matrix preallocation algorithm
    \end{itemize}
    %
  }
}

\IfSubStr*{\jobname}{vitae}{
  \def\coop{
    %
    Built extendable code in collaboration with with \hong~, \lois~, and
    Lawrence Livermore National Laboratory physicists by adding robust
    \petsc~solvers to \bout. This code has been successfully integrated into
    \bout~and version 0.7 incorporates my work.
    %
  }
}{
}

\IfSubStr*{\jobname}{vitae}{
  \def\grad{
    %
    Assisted \blaise~by writing code, providing proofs, building and
    maintaining a cluster, and helping with class work. One graduate level
    class dealt with elliptic solvers ($\Delta u = 0$ for convex domains) and
    made use of finite element code we'd developed the previous summer. In
    class, we extended the code to work on solving edge detection problems
    using the Mumford-Shah functional modeled with phase fields. Other duties
    included helping the \dept's computer support team in the design of a
    Kerberos-based authentication system for use with LDAP, http, and
    Subversion authentication.
    %
  }
}{
  \def\grad{
    %
    \vspace{-4mm}
    \begin{itemize}
      \item Implemented elliptic finite element code
      \item Provided theoretical support for graduate analysis classes
      \item Built and maintained a cluster for three years
      \item Installed a Kerberos authentication system for LDAP and Subversion
    \end{itemize}
    %
  }
}

\IfSubStr*{\jobname}{vitae}{
  \def\givens{
    %
    Worked under \hong~and \lois~to add \petsc~preconditioners to \bout, writing
    new code in C++.
    %
  }
}{
  \def\givens{
    %
    Worked under \hong~and \lois~to add \petsc~preconditioners to \bout, writing
    new code in C++.
    %
  }
}

\IfSubStr*{\jobname}{vitae}{
  \def\teacher{
    %
    Taught around 100 students per semester College Algebra and Trigonometry as
    the teacher of record. In addition to the standard responsibilities of
    teaching (creating lesson plans, grading, etc.), my duties included assisting
    students during weekly mandatory math labs that employed online educational
    software.
    %
  }
}{
  \def\teacher{}
}

\IfSubStr*{\jobname}{vitae}{
  \def\summer{
    %
    Wrote a finite difference and a $P_1$, $P_2$ finite element program for
    solving elliptic equations using MPI and \petsc. The finite element code used
    a custom implementation based on \petsc~distributed arrays. Key concepts
    included structuring data, scatter/gather among processors, and linear
    solvers.
    %
  }
}{
  \def\summer{
    %
    Wrote a finite element program for solving elliptic equations using MPI and
    \petsc.
    %
  }
}

\IfSubStr*{\jobname}{vitae}{
  \def\dotnet{
    %
    Worked with small teams to solve numerous business and technical problems
    with a photography and photo printing company. The main product scheduled
    photographers and allowed them to coordinate shooting a (mostly large) local
    event. Our software eased the challenge of organizing and executing photo
    shoots at multiple, distant events as well as pulling together masses of
    photos, uploading them and posting selected ones automatically to a for sale
    website. Used design patterns, databases, threads, security, web services,
    and xml.
    %
    \newline\newline
    %
    During the immediate aftermath of Hurricane Katrina, our team volunteered to
    help the Red Cross keep track of the thousands of newly arrived storm
    refugees scattered across Baton Rouge. After a 72-hour coding sprint, we
    had a functional program that utilized fuzzy text searching, text indexing,
    and merging databases with high chances of collision.
    %
  }
}{
  \def\dotnet{
   %
    \vspace{-4mm}
    \begin{itemize}
      \item Designed an online scheduler for photographers that allowed them to
      coordinate events and automatically upload photos to a website
      \item Organized a 72-hour coding sprint during Hurricane Katrina to write
      a program to help keep track of refugees. Utilized fuzzy text searching,
      text indexing, and merging databases with high chances of collision
    \end{itemize}
    %
  }
}

\IfSubStr*{\jobname}{vitae}{
  \def\php{
    %
    Designed online web applications with extensive database access using PHP and
    created graphics for web pages. Other duties included configuring hardware,
    installing software, and teaching workshops for educators on Javascript, SQL,
    and Adobe Photoshop 6.
    %
  }
}{
  \def\php{
    %
    Designed web apps with database access using PHP, HTML, and javascript.
    %
  }
}

\IfSubStr*{\jobname}{vitae}{
  \def\viz{
    %
    Worked with a general relativity research group that used OpenGL and CAVE to
    implement C++ libraries to utilize 3D aspects of the SGI ImmersaDesk on
    modeling binary stars.
    %
  }
}{
  \def\viz{
    %
    Worked with a general relativity research group to visualize binary star
    simulations with OpenGL on a
    \href{http://en.wikipedia.org/wiki/Cave_automatic_virtual_environment}{CAVE}.
    %
  }
}

\IfSubStr*{\jobname}{vitae}{
  \def\dynweb{
    %
    Designed and implemented web-based programs for students and faculty,
    utilizing server side languages such as PHP, Java Server Pages, and Perl. My
    most important focus was completing my area of ``Distinction,'' namely an
    online application to register students for seminars.
    %
  }
}{
  \def\dynweb{}
}

\IfSubStr*{\jobname}{vitae}{
  \def\staticweb{
    %
    Created web pages using HTML, CSS, and JavaScript for faculty. Other duties
    included graphic design and knowledge of web server maintenance.
    %
  }
}{
  \def\staticweb{}
}

\IfSubStr*{\jobname}{vitae}{
  \def\experience{
    {06-09 2013}/{Google Summer of Code Student}/{Mercurial Project}/{\chicago}/{\noexpand\gsoc},
    {2010--2013}/{Predoctorate Research Assistant}/{\anl}/{\chicago}/{\noexpand\predoc},
    {06-09 2009}/{Student Co-op}/{\anl}/{\chicago}/{\noexpand\coop},
    {2006--2009}/{Graduate Assistant}/{\lsu}/{\br}/{\noexpand\grad},
    {06-09 2008}/{Givens Internship}/{\anl}/{\chicago}/{\noexpand\givens},
    {2007--2008}/{Undergraduate Math Teacher}/{\lsu}/{\br}/{\noexpand\teacher},
    {06-08 2006}/{Research Programmer with \blaise}/{\lsu}/{\br}/{\noexpand\summer},
    {2003--2006}/{.NET Programmer}/{Velocity Squared, LLC}/{\br}/{\noexpand\dotnet},
    {2001--2002}/{PHP Programmer}/{Louisiana Department of Education}/{\br}/{\noexpand\php},
    {2001--2002}/{Data Visualization Programmer with J.~Tohline}/{\lsu}/{\br}/{\noexpand\viz},
    {2000--2001}/{Dynamic Web Programmer}/{\lsmsa}/{\nat}/{\noexpand\dynweb},
    {1999--2000}/{Web Programmer}/{\lsmsa}/{\nat}/{\noexpand\staticweb}
  }
}{
  \def\experience{
    {06-09 2013}/{Google Summer of Code Student}/{Mercurial Project}/{\chicago}/{\noexpand\gsoc},
    {2010--2013}/{Predoctorate Research Assistant}/{\anl}/{\chicago}/{\noexpand\predoc},
    {2006--2009}/{Graduate Assistant}/{\lsu}/{\br}/{\noexpand\grad},
    {2003--2006}/{.NET Programmer}/{Velocity Squared, LLC}/{\br}/{\noexpand\dotnet},
  }
}

%%% Local Variables:
%%% mode: latex
%%% TeX-master: "vitae"
%%% End:
