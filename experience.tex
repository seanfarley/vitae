\newcommand\myindent{\setlength\parindent{18pt}}

\IfSubStr*{\jobname}{vitae}{
  \def\osi{
    %
    \myindent \emph{Designed and implemented a parser to generate an abstract
      syntax tree in C\#}.
    \\
    I diagnosed a long--standing bug with incorrect parsing behavior in a
    customer plugin. I fixed the bug by implementing a tokenizer to correctly
    give the full AST. I worked with Microsoft solutions: Team Foundation
    Server, Visual Studio, and PowerShell.
    %
  }
}{
  \def\osi{
    %
    \vspace{-4mm}
    \begin{itemize}
      \item Implemented a parser to generate a full abstract syntax tree
    \end{itemize}
    %
  }
}

\IfSubStr*{\jobname}{vitae}{
  \def\fbcontractor{
    %
    \myindent \emph{Designed a namespace API to allow for managing tags,
      bookmarks, and branches}.
    \\
    One of the biggest missing features in Mercurial is the lack of remote
    branches. Due to this hot topic in the community, this task took many
    months of planning and gathering feedback. After two sprints and many
    rounds of code review, the framework was accepted and released in v3.3.
    Building on top of this namespace API, I drafted a remote branches
    extension. This is a work in progress but has already received much
    positive feedback from the community.
    %
  }
}{
  \def\fbcontractor{
    %
    \vspace{-4mm}
    \begin{itemize}
      \item Designed and wrote namespaces API that allows any extension to
        cleanly operate with bookmarks, tag, and branches.
      \item Created an extension using namespaces api to track remote branches
        and bookmarks
    \end{itemize}
    %
  }
}

\IfSubStr*{\jobname}{vitae}{
  \def\gsoc{
    %
    \myindent \emph{Engineered deep refactor of the internal representation of
      changesets in memory}.
    \\
    This work opened the way for improving the recording feature (arbitrary
    selection of hunks upon committing) and in-memory rebasing for
    \href{http://mercurial.selenic.com}{Mercurial}. The biggest hurdle of this
    project was tackling the refactoring needed of the legacy
    \href{http://selenic.com/hg/file/c38c3fdc8b93/mercurial/localrepo.py\#l1458}{\texttt{status}}
    and
    \href{http://selenic.com/hg/file/c38c3fdc8b93/mercurial/localrepo.py\#l1149}{\texttt{commit}}
    functions. This summer program provided a great educational process of
    learning the depths of internal Mercurial code, specifically the
    \href{http://selenic.com/hg/file/tip/mercurial/context.py}{\texttt{context}},
    \href{http://selenic.com/hg/file/tip/mercurial/manifest.py}{\texttt{manifest}},
    and
    \href{http://selenic.com/hg/file/tip/mercurial/localrepo.py}{\texttt{localrepo}}
    objects. This program was successfully completed with over 127 patches
    accepted (most students averaged 20 patches).
    %
  }
}{
  \def\gsoc{
    %
    \vspace{-4mm}
    \begin{itemize}
      \item Cleaned up legacy code to provide clear reference points for file status
      \item Augmented file-merging framework for in-memory changes
      \item Successful evaluation with over 127 patches accepted (average 20)
    \end{itemize}
    %
  }
}

\IfSubStr*{\jobname}{vitae}{
  \def\predoc{
    %
    \myindent \emph{Wrote a paper on \bout~(BOUndary Turbulence), a parallel
      edge turbulence framework.}
    \\
    I added framework for physics-based preconditioning, advanced nonlinear
    solvers, advanced time-stepping methods, new mesh capabilities, and
    improved the build system for better software organization. The majority of
    the work involved improving meshing capabilities to add matrix
    preallocation for the non-contiguous branch-cuts. In addition, robust and
    novel IMEX (implicit-explicit) time-stepping methods were added giving the
    project competitive performance results. Version 1.0 incorporates my work.
    %
  }
}{
  \def\predoc{
    %
    \vspace{-4mm}
    \begin{itemize}
      \item Published paper on \bout, a library for plasma nuclear fusion simulation
      \item Developed advanced algorithms for nonlinear problems (NGMRES)
      \item Integrated robust time-stepping methods for \petsc~(IMEX)
      \item Added framework for new mesh capabilities
    \end{itemize}
    %
  }
}

\IfSubStr*{\jobname}{vitae}{
  \def\coop{
    %
    \myindent \emph{Added \petsc~solvers to \bout.}
    \\
    Built extendable code in collaboration with with \hong~, \lois~, and
    Lawrence Livermore National Laboratory physicists by adding robust
    \petsc~solvers to \bout. This code has been successfully integrated into
    \bout~and version 0.7 incorporates my work.
    %
  }
}{
}

\IfSubStr*{\jobname}{vitae}{
  \def\grad{
    %
    \myindent \emph{Wrote research code to study fracture mechanics.}
    \\
    Assisted \blaise~by writing code, providing proofs, building and
    maintaining a cluster, and helping with class work. One graduate level
    class dealt with elliptic solvers ($\Delta u = 0$ for convex domains) and
    made use of finite element code we'd developed the previous summer. In
    class, we extended the code to work on solving edge detection problems
    using the Mumford-Shah functional modeled with phase fields. Other duties
    included helping the \dept's computer support team in the design of a
    Kerberos-based authentication system for use with LDAP, http, and
    Subversion authentication.
    %
  }
}{
  \def\grad{
    %
    \vspace{-4mm}
    \begin{itemize}
      \item Implemented finite element code ($P_1, P_2$)
      \item Provided theoretical support for graduate analysis classes
      \item Built and maintained a computing cluster for three years for a
        research group using \mpi
    \end{itemize}
    %
  }
}

\IfSubStr*{\jobname}{vitae}{
  \def\givens{
    %
    Worked under \hong~and \lois~to add \petsc~preconditioners to \bout, writing
    new code in C++.
    %
  }
}{
  \def\givens{
    %
    Worked under \hong~and \lois~to add \petsc~preconditioners to \bout, writing
    new code in C++.
    %
  }
}

\IfSubStr*{\jobname}{vitae}{
  \def\teacher{
    %
    \myindent \emph{Taught college algebra and trigonometry to $\sim$100
      students per semester.}
    \\
    In addition to the standard responsibilities of teaching (creating lesson
    plans, grading, etc.), my duties included assisting students during weekly
    mandatory math labs that employed online educational software.
    %
  }
}{
  \def\teacher{}
}

\IfSubStr*{\jobname}{vitae}{
  \def\summer{
    %
    \myindent \emph{Conceptualized a numerical program for solving elliptic
      equations using MPI and \petsc.}
    \\
    The finite element code used a custom implementation of $P_1$ and $P_2$
    elements based on \petsc~distributed arrays. Key concepts included
    structuring data, scatter/gather among processors, and linear solvers.
    %
  }
}{
  \def\summer{
    %
    Wrote a finite element program for solving elliptic equations using MPI and
    \petsc.
    %
  }
}

\IfSubStr*{\jobname}{vitae}{
  \def\dotnet{
    %
    \myindent \emph{Designed an online scheduler for photographers that allowed
      them to coordinate events.}
    \\
    Worked with small teams to solve numerous business and technical problems
    with a photography and photo printing company. The main product scheduled
    photographers and allowed them to coordinate shooting a (mostly large)
    local event. Our software eased the challenge of organizing and executing
    photo shoots at multiple, distant events as well as pulling together masses
    of photos, uploading them and posting selected ones automatically to a for
    sale website. Used design patterns, databases, threads, security, web
    services, and xml.
    %
    \newline\newline
    %
    During the immediate aftermath of Hurricane Katrina, our team volunteered to
    help the Red Cross keep track of the thousands of newly arrived storm
    refugees scattered across Baton Rouge. After a 72-hour coding sprint, we
    had a functional program that utilized fuzzy text searching, text indexing,
    and merging databases with high chances of collision.
    %
  }
}{
  \def\dotnet{
   %
    \vspace{-4mm}
    \begin{itemize}
      \item Designed a Microsoft SQL-based scheduler with C\# for photographers
      that allowed them to simplify their workflow
      \item Organized a 72-hour coding sprint during Hurricane Katrina to write a
        refugee tracking program
      \item Utilized fuzzy text searching, text indexing, and merging databases
        with high chances of collision
    \end{itemize}
    %
  }
}

\IfSubStr*{\jobname}{vitae}{
  \def\php{
    %
    Designed online web applications with extensive database access using PHP and
    created graphics for web pages. Other duties included configuring hardware,
    installing software, and teaching workshops for educators on Javascript, SQL,
    and Adobe Photoshop 6.
    %
  }
}{
  \def\php{
    %
    Designed web apps with database access using PHP, HTML, and javascript.
    %
  }
}

\IfSubStr*{\jobname}{vitae}{
  \def\viz{
    %
    Worked with a general relativity research group that used OpenGL and CAVE to
    implement C++ libraries to utilize 3D aspects of the SGI ImmersaDesk on
    modeling binary stars.
    %
  }
}{
  \def\viz{
    %
    Worked with a general relativity research group to visualize binary star
    simulations with OpenGL on a
    \href{http://en.wikipedia.org/wiki/Cave_automatic_virtual_environment}{CAVE}.
    %
  }
}

\IfSubStr*{\jobname}{vitae}{
  \def\dynweb{
    %
    Designed and implemented web-based programs for students and faculty,
    utilizing server side languages such as PHP, Java Server Pages, and Perl. My
    most important focus was completing my area of ``Distinction,'' namely an
    online application to register students for seminars.
    %
  }
}{
  \def\dynweb{}
}

\IfSubStr*{\jobname}{vitae}{
  \def\staticweb{
    %
    Created web pages using HTML, CSS, and JavaScript for faculty. Other duties
    included graphic design and knowledge of web server maintenance.
    %
  }
}{
  \def\staticweb{}
}

\IfSubStr*{\jobname}{vitae}{
  \def\experience{
    {2014--now}/{Software Developer}/{OSIsoft, LLC}/{\sanfran}/{\noexpand\osi},
    {04-09 2014}/{Facebook Contractor}/{Mercurial Project}/{\chicago}/{\noexpand\fbcontractor},
    {06-09 2013}/{Google Summer of Code Student}/{Mercurial Project}/{\chicago}/{\noexpand\gsoc},
    {2010--2013}/{Graduate Research Assistant}/{\anl}/{\chicago}/{\noexpand\predoc},
    {06-09 2009}/{Student Co-op}/{\anl}/{\chicago}/{\noexpand\coop},
    {2006--2009}/{Graduate Assistant}/{\lsu}/{\br}/{\noexpand\grad},
    {06-09 2008}/{Givens Internship}/{\anl}/{\chicago}/{\noexpand\givens},
    {2007--2008}/{Undergraduate Math Teacher}/{\lsu}/{\br}/{\noexpand\teacher},
    {06-08 2006}/{Research Programmer with \blaise}/{\lsu}/{\br}/{\noexpand\summer},
    {2003--2006}/{.NET Programmer}/{Velocity Squared, LLC}/{\br}/{\noexpand\dotnet},
    {2001--2002}/{PHP Programmer}/{Louisiana Department of Education}/{\br}/{\noexpand\php},
    {2001--2002}/{Data Visualization Programmer with J.~Tohline}/{\lsu}/{\br}/{\noexpand\viz},
    {2000--2001}/{Dynamic Web Programmer}/{\lsmsa}/{\nat}/{\noexpand\dynweb},
    {1999--2000}/{Web Programmer}/{\lsmsa}/{\nat}/{\noexpand\staticweb}
  }
}{
  \def\experience{
    {2014--now}/{Software Developer}/{OSIsoft, LLC}/{\sanfran}/{\noexpand\osi},
    {04-09 2014}/{Facebook Contractor}/{Mercurial Project}/{\chicago}/{\noexpand\fbcontractor},
    {06-09 2013}/{Google Summer of Code Student}/{Mercurial Project}/{\chicago}/{\noexpand\gsoc},
    {2010--2013}/{Graduate Research Assistant}/{\anl}/{\chicago}/{\noexpand\predoc},
    {2006--2009}/{Graduate Assistant}/{\lsu}/{\br}/{\noexpand\grad},
  }
}

%%% Local Variables:
%%% mode: latex
%%% TeX-master: "resume"
%%% End:
